\documentclass{article}

\title{Reading08, A Tale of Love, Loss, and LaTeX}
\date{3/16/16}
\author{Owen Phelan}

\begin{document}
	\pagenumbering{gobble}
	\maketitle
%------------------------------------------------------------------------------
	\section{overview}
	\begin{verbatim}
For this assignment I created the scripts roll_dice.sh, experiment.sh, and histogram.plt.

roll_dice.sh - a program which accepts the number of sides for the dice (DEFAULT=6), and 
number of rolls (DEFAULT=10). it then ouputs N random roll results, of a N sided die.

experiment.sh - uses roll_dice.sh to output the result of rolling a standard dice 1000 
times to a file named results.dat

histogram.plt - uses the gnuplot command, along with the results.dat file, to output the 
results to a histogram graph.
	\end{verbatim}
%------------------------------------------------------------------------------
	\section{RollingDice}
	\begin{verbatim}
This shell script uses standard methods to accept user arguments for s (number of sides), 
and r (number of rolls).vThe shell script works via a number of nested loops
1. The script outputs all the possible rolls in order. 
2. These outputs are then shuffled, so that the first output is a random one of the 
possibilities, using shuf
3. This first output is then taken by itself, using head
4. This is repeated for the number of rolls requested, so that there are ROLL# random 
possibilities as output.
	\end{verbatim}
%------------------------------------------------------------------------------
	\section{Experiment}
	\begin{verbatim}
This shell script uses the exact same code as roll_dice.sh. It then uses awk & associative 
arrays to keep track.
1. A random list of output is sent from the roll_dice code
2. This output is then passed to awk, which keeps track using an array called 'count'
3. Each time a roll result is passed to awk, count[ - the given roll result - ] is 
incremented.
4. Awk reaches the END of it's parsing, and then prints all the possible results, and the 
number of their occurances
separated by a tap.
5. This input is redirected to the file "results.dat"
	\end{verbatim}
%------------------------------------------------------------------------------
	\section{Results}
	\begin{figure}[h!]
	\centering
	%\includegraphics{results.png}
	%\caption{Results Histogram}
	\end{figure}
	\newpage
%------------------------------------------------------------------------------
\end{document}