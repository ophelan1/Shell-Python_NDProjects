\documentclass{article}

\title{Project 01}
\date{4/21/16}
\author{Owen Phelan, ophelan1@nd.edu}

\begin{document}
	\pagenumbering{gobble}
	\maketitle
%------------------------------------------------------------------------------
	\section{Summary}
	\begin{verbatim}
To analyze the data I used dataProcessor.sh, which uses awk to separate the 
data by column and calculate the totals. The result of this data parsing is 
was disturbing. In the past 4 years, the demographics of Notre Dame students 
has not changed. Women have made up around 22% of every class, every year. 
The number of African American students has also not changed. This holds for
every other demographic as well. 

It is hard to interpret this data without first analyzing the demographics
of Notre Dame undergrads as a whole. However, in a school that is more or 
less half male and half female, the gender numbers at least are clearly 
unbalanced. 

	\end{verbatim}
%------------------------------------------------------------------------------
	\section{Latency}
	\begin{verbatim}
As of right now, everything in my project is working except for the thor.py script and the SIGCHLD handling. I cannot for the life of me 
figure out why SIGCHLD is not a signal that is recognized by my computer. As for the thor.py script, I still need to change up parsing the 
request url so that I can get an accurate path. I did this project alone because my schedule has been so crazy lately that I didn't feel comfortable asking my friends to work with me as late as I usually work.

	\end{verbatim}
	\newpage
%------------------------------------------------------------------------------
	\section*{Throughout}
	\begin{verbatim}
I tested the throughout return speed by using thor so send 200 requests to the addresses corresponding the files. These are the averages
I obtained. There is not much of a difference as far as I can tell. I believe there may be an issue with my Thor program.

	\end{verbatim}
	\begin{table}[h!]
	    \centering
	    \begin{tabular}{c|c|c}
	    Type	& 	 Non-Forking &  Forking\\
	    \hline
	    dir		& 	.01 sec 			& 	.02 sec	\\
	    file		& 	.01 sec 			& 	.02 sec	\\
	    script	& 	.01 sec			& 	.02 sec 	\\
	    \end{tabular}
	    \caption{Test Results}
	    \label{tbl:Data Results}
	\end{table}
%------------------------------------------------------------------------------
	\section*{Analysis}
	\begin{verbatim}
As there is an issue with my thor.py script, I feel there is no reason to investigate this test. Though if I had been able to, I would have sent
200 processes in 1 request using thor. I then would compare the time differences. Almost inevitably the forking mode should be faster. 

	\end{verbatim}
	\end{table}
%------------------------------------------------------------------------------
	\section{Conclusion}
	\begin{verbatim}
I thought this project was really cool, because it gave me a practical understanding of what is happening when I open a web browser. 
It's fun understanding 'what's under the hood' in a sense. Before this project I had no idea what sockets were, how they functioned, or what was happening every time I made a request to one. Internet security is something I'd like to look into, because it seems like something that would fascinate me, and knowing what is actually happening every time a server is contacted is a fun look into what that kind of work may look like. 
						
	\end{verbatim}
%------------------------------------------------------------------------------
\end{document}