\documentclass{article}

\title{Homework 06}
\date{3/18/16}
\author{Owen Phelan, phelan.eoin@gmail.com}

\begin{document}
	\pagenumbering{gobble}
	\maketitle
%------------------------------------------------------------------------------
	\section{Overview}
	\begin{verbatim}
To analyze the data I used dataProcessor.sh, which uses awk to separate the 
data by column and calculate the totals. The result of this data parsing is 
was disturbing. In the past 4 years, the demographics of Notre Dame students 
has not changed. Women have made up around 22% of every class, every year. 
The number of African American students has also not changed. This holds for
every other demographic as well. 

It is hard to interpret this data without first analyzing the demographics
of Notre Dame undergrads as a whole. However, in a school that is more or 
less half male and half female, the gender numbers at least are clearly 
unbalanced. 

	\end{verbatim}
%------------------------------------------------------------------------------
	\section{Methadology}
	\begin{verbatim}
I got the data using awk. 
First, an awk grabs individual years (columns 1 & 2, 3 & 4, 5 & 6, 7 & 8)
This is done twice, so that the results can be piped to two different awks
The two different awks that it is piped to are used to print the totals, 
both for the genders and for the ethnicities. Once everything was totaled 
and computed, it was easy to enter the data into the table.

	\end{verbatim}
	\newpage
%------------------------------------------------------------------------------
	\section*{Analysis}
	\begin{table}[h!]
	    \centering
	    \begin{tabular}{c|c|c|c|c|c|c}
	    Year	& 	 Female &  Male & White & Black & Asian & Hispanic\\
	    \hline
	    2013		& 	14 	& 	49 	& 	43 	& 	3 	& 	7 	& 	7\\
	    2014		& 	12 	& 	44 	& 	43 	& 	2 	& 	5 	& 	4\\
	    2015		& 	16 	& 	58 	& 	47 	& 	4 	& 	9 	& 	10\\
	    2016		& 	19 	&	60 	& 	53 	&	1 	& 	9 	& 	9\\
	    \end{tabular}
	    \caption{Data Results}
	    \label{tbl:Data Results}
	\end{table}
%------------------------------------------------------------------------------
	\section{Discussion}
	\begin{verbatim}
    This assignment is a really challenging one for me. One the one hand,
I care very deeply about gender and race equality. I took the white privilege
seminar last year, and attended the white privilege conference in Lousiville 
that spring. I  have spent two summers teaching in west Philadelphia, and have
experienced first hand how minorities can be taught that certain futures are
simply not open to them. 
   All this is to say that, usually, I am the first person in line calling
for an increase in diversity and gender equality. This is the reason this 
assignment is so difficult to answer. I am not sure that diversity is necessary 
in a software engineering environment. 
   The reason having a diverse team - despite being indicative of an accepting
culture - is important for obvious reasons. When you have different people from
different backgrounds helping to solve a problem, you will get different 
input and opinions. You will be better able to come up with a solution, because
there will be a diverse set of possible solutions available. When you have a 
lack of diversity, you have a lack of opinion. However, I'm not sure how this
can possibly be applicable to computer science. Whether you have a team with 
three white men, or a white man a black man and a Hispanic woman, there is no 
difference in the tam's ability to code something. A diverse team is necessary
to think of the best programs and services and design. A diverse team is not 
necessary to actually implement these designs. 
   Take the example of a plumbing company. If you have a team made up entirely
of men, the company will not be able to design good products. Can you imagine a 
bathroom designed entirely by men? It would be a fiasco? That being said,
once the bathroom is designed, actually fitting the pipes together is not 
something that requires diverse input. Either your plumber can assemble pipes
correctly, and is a good plumber, or is not. Their gender, race, etc. has no 
impact on their ability, and should not enter the equation when they are 
applying for a job.
   I believe Google's statistics are a fair representation of this belief.
In non-coding jobs, their gender employment rate is more or less equal. However,
for jobs that actually require coding, their employment rate is skewed heavily
in the male direction. I do not believe this is a problem with Google. 
Especially because the application process for software engineers is normally
blind (coding challenge sent via email) until later rounds. Google probably
employs much more men, because there are many more competitive male coders
then female coders.
   The issue then is not how to make computer science more diverse, in my 
experience the tech industry is one of the most accepting communities there
is. Rather, the issue is "What is is about our society that attracts men to 
computer science, and not women".
   This directly relates to the computer science department at ND, but it 
makes the job of our computer science department much more difficult. First of
all, we are a school with absolutely horrifying gender and race relations. This
is of course a personal opinion, but one that I truly believe. Notre Dame 
operates on gender expectations which spring from outdated Catholicism (I will
cite the fact that disciplinary action in the school tends to function on the
principle that women should live in cloisters but "boys will be boys"). In 
addition, the majority of privileged Irish catholics in this country are white,
not surprisingly. This means that a lot of Notre Dame kids have never had a
minority as a friend, and a lot of Notre Dame kids have never had friends 
without a lot of money. All this makes for a school of educated men and women,
who SAY that they care about gender equality and racial exception, but who 
have never had the opportunity to develop an actual empathy for those 
demographics they claim to be defending. This is not their fault, I honestly
believe that they do not know any better (during the white privilege seminar 
I had a number of male students tell me that it's no better to be a man than 
a woman in America, and no better to be white then black.... which is simply
incorrect).
   All that being said, how can we expect women who do not follow typical 
gender norms (woman who play computer games, women who like computer science), 
to want to attend a school like Notre Dame, where outdated gender norms are 
socially enforced? How can we expect minorities to want to attend here? And 
once these women and minorities DO come to ND, and if they decide to do 
computer science, what can Notre Dame do to include them further? As far
as I can tell, the issue is the environment at Notre Dame, the issue is 
the environment in America. 						
	\end{verbatim}
%------------------------------------------------------------------------------
\end{document}